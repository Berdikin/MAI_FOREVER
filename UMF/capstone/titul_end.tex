

\documentclass[a4paper]{article}
\usepackage[14pt]{extsizes} % для того чтобы задать нестандартный 14-ый размер шрифта
\usepackage[utf8]{inputenc}
\usepackage[russian]{babel}
\usepackage{setspace,amsmath}
\usepackage[left=20mm, top=15mm, right=15mm, bottom=15mm, nohead, footskip=10mm]{geometry} % настройки полей документа
\usepackage{listings} 

\begin{document} % начало документа
\begin{titlepage}
\begin{center}
\bfseries

{\Large Московский авиационный институт\\ (национальный исследовательский университет)

}

\vspace{48pt}

{\large Факультет информационных технологий и прикладной математики
}

\vspace{36pt}


{\large Кафедра вычислительной математики и~программирования

}


\vspace{48pt}

Курсовой проект по курсу \enquote{<<Уравнения математической физики>>}

\end{center}

\vspace{72pt}

\begin{flushright}
\begin{tabular}{rl}
Студент: & Т.\,А. Бердикин \\
Преподаватель: & С.\,А. Колесник \\
Группа: & М8О-307Б-18 \\
Вариант: & 8 \\
Дата: & \\
Оценка: & \\
Подпись: & \\
\end{tabular}
\end{flushright}

\vfill

\begin{center}
\bfseries
Москва, \the\year
\end{center}
\end{titlepage}
 

\newpage
\begin{thebibliography}{99}
\bibitem{Tikhonov}
Тихонов\,А.\,Н, Самарский\,А.\,А.
{\itshape Изд. 5, стереотипное.} --- М.: \enquote{Наука}, 1977.

\bibitem{Budak}
Будак\,Б.\,М, Тихонов\,А.\,Н, Самарский\,А.\,А.
{\itshape Сборник задач по математической физике.} --- М.: \enquote{Наука}, 1980.

\bibitem{Bogolubov}
Боголюбов\,А.\,Н, Кравцов\,В.\,В.
{\itshape Задачи по математической физике} --- М.: \enquote{Издательство Московского университета}, 1998.


\end{thebibliography}
\pagebreak


\end{document}  % КОНЕЦ ДОКУМЕНТА !